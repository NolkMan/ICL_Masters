\documentclass[11]{article}   % list options between brackets

\usepackage{graphicx}
\usepackage{subcaption}
\usepackage{float}
\usepackage[a4paper,  margin=1in]{geometry}

\begin{document}

\title{COMP70004 Advanced Computer Security\\Coursework 2 Report} 
\author{Michal Glinski (mjg222)}    

\maketitle

\section{Content Security Policy}
\subsection{Overview}
Content Security Policy(CSP) was introduced as a protection layer with an aim to prevent Cross Site Scripting(XSS) and DOM injection based attacks directly in the user browser.
W3 organization recommends using CSP as a defence-in-depth tool, which can help prevent basic attacks and reduce the harm caused by malicious users.
% https://www.w3.org/TR/CSP3/
It works by server providing the user a whitelist of allowed endpoints from which sources can be retreived.
In CSP sources can be additionally separated by content type and usage; scripts, stylesheets and images have their own directives. 
This allows for more fine grained control over loaded content

CSP is constantly being developed with currently 3rd version being in draft, while parts of it are already being implemented into modern browsers.
With CSP3 \texttt{scirpt-src} directive has been further split into \texttt{script-src-elem} and \texttt{script-src-attr} directives which seperate javascript coming from \texttt{<script>} markup tags and attributes such as \texttt{onClick} event attribute.

\subsection{Implementing CSP}

Adding nonces may not increase security

\subsection{Bypassing CSP}
Even though CSP can prevent many basic attacks it may be unsuccessful at blocking more advanced exploits.
A study by ?? \cite{??} done in 2016 showcases many ways in which CSP may be ineffective. 
It also performs a comprahensive analisys on CSP directives which concludes that most of them are trivially bypassable through methods they have introduced.



\subsection{Reporting}

\end{document}
